\chapter*{Úvod}
\addcontentsline{toc}{chapter}{Úvod}

Židovské město ve Třebíči jsem si vybral pro jeho unikátnost.
Na rozdíl od většiny bylo už při svém vzniku přímo v centru města.
Nehledě na to, že řada budov zůstala zachována ve velmi dobrém stavu.
Zároveň budovy samotné jsou architektonicky velice zajímavé, po tvorbě ghetta se město nemohlo dál rozšiřovat a vznikla tak velmi hustá, průchody protkaná, zástavba.
Třebíč je místem, které jsem sám dvakrát navštívil a velice mě zaujalo jako malebné moravské městečko.
V neposlední řadě mě lákal koncept tématu této práce -- zmapovat staletí dějin na jednom malém místě.

Cílem práce je prozkoumat jak dějiny Židovského města, tak dějiny Třebíče.
Poté se zaměřit na unikátní část města -- Židé a křesťané žili v těsném soužití, denně se zde střetávaly dva odlišné světy.
Nakonec práce projde historii některých vybraných staveb Židovského města jako třeba dvě synagogy, jež se v Třebíči nachází.
