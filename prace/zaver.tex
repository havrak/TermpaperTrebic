\chapter*{Závěr}
\addcontentsline{toc}{chapter}{Závěr}

Cíl práce se i přes problémy podařilo splnit.
Bohužel se mi díky epidemii CoVid--19 nepodařilo město navštívit a získat tak mé osobní fotografie.
Ty však tvoří jen malou část práce.
Během nočního přejezdu jsem ve městě zastavil a čtvrť alespoň prošel.

Jediný problém ke zpracování představovalo období dějin města od revolučního roku 1848 po současnost.
Bohužel v tomto časovém rozsahu nejsou ani publikace ohledně samotného Třebíče.
Byla sice plánována kniha Třebíč: Dějiny města III., k jejímu vytvoření snad nikdy nedošlo, či vzniklo jen pár výtisků.
Obecně nedostatek středoškolskému studentu přístupných pramenů komplikuje popsat dějiny židovské komunity v Třebíči podrobně.

I tak si myslím, že je výsledkem této práce obstojný souhrnný pohled na historii židovské obce ve Třebíči.
Sám jsem díky této práci získal hluboký náhled do dějin nejen samotného Židovského města, ale i celého Třebíče a Moravy.
Pevně doufám, že se tohoto tématu ujme schopný historik a celé dějiny zmapuje podrobně od začátku do konce.
