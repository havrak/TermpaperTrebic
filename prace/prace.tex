%%% Hlavní soubor. Zde se definují základní parametry a odkazuje se na ostatní části. %%%

%% Verze pro jednostranný tisk:
% Okraje: levý 40mm, pravý 25mm, horní a dolní 25mm
% (ale pozor, LaTeX si sám přidává 1in)
\documentclass[a4paper,oneside,12p]{report}
%\documentclass[a4paper,oneside,12p,twoside]{report}
\setlength\textwidth{155mm}
\setlength\textheight{247mm}
%\setlength\oddsidemargin{00mm}
\setlength\topmargin{-10mm}
\setlength\headheight{0mm}
\setlength\oddsidemargin{05mm}
\setlength\evensidemargin{05mm}
\let\openright=\clearpage


%% Vytváříme PDF/A-2u
\usepackage[a-2u]{pdfx}

%% Přepneme na českou sazbu a fonty Latin Modern
\usepackage[czech]{babel}
\usepackage{lmodern}
\usepackage[T1]{fontenc}
\usepackage{textcomp}

%% Použité kódování znaků: obvykle latin2, cp1250 nebo utf8:
\usepackage[utf8]{inputenc}

%%% Další užitečné balíčky (jsou součástí běžných distribucí LaTeXu)
\usepackage{amsmath}        % rozšíření pro sazbu matematiky
\usepackage{amsfonts}       % matematické fonty
\usepackage{amsthm}         % sazba vět, definic apod.
\usepackage{bbding}         % balíček s nejrůznějšími symboly
			   										% (čtverečky, hvězdičky, tužtičky, nůžtičky, ...)
\usepackage{bm}             % tučné symboly (příkaz \bm)
\usepackage{graphicx}       % vkládání obrázků
\usepackage{fancyhdr}				% možnost slylizovat záhlaví
\usepackage{fancyvrb}       % vylepšené prostředí pro strojové písmo
\usepackage{indentfirst}    % zavede odsazení 1. odstavce kapitoly
\usepackage[nottoc]{tocbibind} % zajistí přidání seznamu literatury,
\usepackage{icomma}         % inteligetní čárka v matematickém módu
\usepackage{dcolumn}        % lepší zarovnání sloupců v tabulkách
\usepackage{booktabs}       % lepší vodorovné linky v tabulkách
\usepackage{paralist}       % lepší enumerate a itemize
\usepackage{caption}				%	popisky
\usepackage{dirtree}				% strom souborů
\usepackage[bottom]{footmisc}  % poznámky pod čarou vespod
\usepackage{bibentry}
\nobibliography*

\usepackage{color}
\usepackage{forest}					% na rodokmeny
\usepackage{natbib}
%\setbibentrystyle{author}

\definecolor{pblue}{rgb}{0.13,0.13,1}
\definecolor{pgreen}{rgb}{0,0.5,0}
\definecolor{pred}{rgb}{0.9,0,0}
\definecolor{pgrey}{rgb}{0.46,0.45,0.48}

%%% Údaje o práci

\def\NazevSkoly{Gymnázium, Praha 6, Arabská 14}
% Název oboru včetně počátečního 'Obor'.
\def\NazevOboru{Dějepis}

% Název práce v jazyce práce (přesně podle zadání)
\def\NazevPrace{Židovké město v Třebíči}

% Název práce v angličtině
\def\NazevPraceEN{Jewish town in Třebíč}

% Název práce v němčině
\def\NazevPraceDE{Jüdische Stadt in Třebíč}

% Jméno auto
\def\AutorPrace{Havránek Kryštof 2.E}

% Rok odevzdání
\def\RokOdevzdani{2020}
% Měsíc odevzdání
\def\MesicOdevzdani{Květen}

% Vedoucí práce: Jméno a příjmení s~tituly
\def\Vedouci{PhDr. Lenka Dvořáková }

% Nepovinné poděkování (vedoucímu práce, konzultantovi, tomu, kdo
% zapůjčil software, literaturu apod.)
\def\Podekovani{%
\textbf{Poděkování}

TODO

PhDr. Lenka Dvořáková

Milan Krčmář (zjisit si titul!)
}

% Abstrakt (doporučený rozsah cca 80-200 slov; nejedná se o zadání práce)
\def\Abstrakt{%
TODO
}
\def\AbstraktEN{%
TODO
}
\def\AbstraktDE{%
TODO
}

% 3 až 5 klíčových slov (doporučeno), každé uzavřeno ve složených závorkách
\def\KlicovaSlova{%
{klíčová} {slova}
}
\def\KlicovaSlovaEN{%
{key} {words}
}


%% Balíček hyperref, kterým jdou vyrábět klikací odkazy v PDF,
%% ale hlavně ho používáme k uložení metadat do PDF (včetně obsahu).
%% Většinu nastavítek přednastaví balíček pdfx.
\hypersetup{unicode}
\hypersetup{breaklinks=true}

%% Definice různých užitečných maker (viz popis uvnitř souboru)
\include{makra}

%% Titulní strana a různé povinné informační strany
\fancypagestyle{plain}{
\fancyhead[C]{}
\fancyhead[L]{Ročníková práce - Gymnázium, Praha 6, Arabská 14}
\fancyhead[R]{\textbf{Židovské město v Třebíči}}
\fancyfoot[L]{Vypracoval: Havránek Kryštof 2.E (Programování)}
\fancyfoot[C]{}
\fancyfoot[R]{\thepage}
\renewcommand{\headrulewidth}{0.4pt}
\renewcommand{\footrulewidth}{0.4pt}
}



\begin{document}

\include{titulka}

\tableofcontents
\newpage


\chapter*{Úvod}
\addcontentsline{toc}{chapter}{Úvod}

Židovské město ve Třebíči jsem si vybral pro jeho unikátnost.
Na rozdíl od většiny leželo židovské osídlení už při svém vzniku přímo v centru města.
Navíc řada budov zůstala zachována ve velmi dobrém stavu.
Zároveň stavby samotné jsou architektonicky velice zajímavé, po tvorbě ghetta se město nemohlo dál rozšiřovat a vznikla tak velmi hustá, průchody protkaná, zástavba.
Třebíč je místem, které jsem sám dvakrát navštívil a velice mě zaujalo jako malebné moravské městečko.
V neposlední řadě mě lákal koncept tématu této práce -- zmapovat staletí dějin jednoho malého místa.

Cílem práce je prozkoumat jak dějiny Židovského města, tak dějiny Třebíče.
Poté se zaměřit na unikátní stránku města -- Židé a křesťané žili v těsném soužití, denně se zde střetávaly dva odlišné světy, jak toto soužití probíhalo.
Nakonec projít historii některých vybraných staveb Židovského města jako třeba dvě synagogy, jež se v Třebíči nachází.


\pagenumbering{arabic}
\setcounter{page}{1}

\chapter{Počátky města Třebíče}

\section{Charakteristika města Třebíče}

Dnes je Třebíč nepříliš velké město v jihovýchodní části kraje Vysočina.
Hrálo však důležitou roli v historii okolí.
O tom svědčí fakt, že má dokonce dvě památky zapsané v seznamu UNESCO, a to Třebíčskou židovskou čtvrť a baziliku svatého Prokopa.
Počátky jeho historie jsou spjaty s benediktýnským klášterem, dnešním zámkem.
Je možné, že na území Třebečského lesa\footnote{území, na kterém byla založena Třebíč} se nacházelo i prehistorické osídlení, o tom však nemáme důkazy\footnote{c. d. \bibentry{Uhlir1978}: 14}.

\section{Vznik kláštera}

Dějiny kláštera se začínají psát koncem 11. století.
Jednalo se o dobu, kdy na českém i moravském území docházelo k řadě konfliktů mezi jednotlivými Přemyslovci a moravskými knížeti. Právě ty vedly následek založení kláštera.
Břetislav II. jako svého následníka jmenoval svého nevlastního syna Bořivoje II., místo brněnského knížete Oldřicha\footnote{ten měl být králem dle rodové tradice Přemyslovců}.
Kvůli této a dalším neshodám uspořádal roku 1099 král Břetislav II. tažení na brněnský hrad, což donutilo Oldřicha a jeho bratra Litolda, jenž spravoval znojemský úděl, k útěku.
Po dvou letech vlády Bořivoje se však Oldřich a Litold opět vrátili na své úděly, využili šance kdy se po smrti Břetislava II. Bořivoj II. musel vrátit do Prahy.

Již při cestě zpět si uvědomili, že západní hranice jejich údělu jsou příliš daleko od hradů v Brně a Znojmě.
Z tohoto důvodu v roce 1101 založili na území Třebečského lesa téměř na hranicích s Přibyslavickem benediktýnský klášter.
Klášter měl sloužit jako předsunuté opevnění a kolonizační centrum.
Již v roce 1101\footnote{c. d. \bibentry{Uhlir1978}: 20} byl postaven kostelík sv. Benedikta, který o tři roky později vysvětil pražský biskup Heřman.
Dále roku 1109\footnote{c. d. \bibentry{Uhlir1978}: 20} vysvětil biskup olomouckým Jan II. samotný klášterní chrám Nanebevztí Pany Marie.
Klášter ihned po založení dostával různá územní darování, a to jak od Oldřicha a Litolda, tak i od jejich nástupců.
Samotný Litold (\CrossOpenShadow 1112) a Oldřich (\CrossOpenShadow 1113) jsou dle slov Tomáše Pěšiny\footnote{Tomáš Pěšiny z Čechorodu -- biskup, dějepisec, autor díla Mars Moravicus} pochováni v klášteře.

\begin{figure}[h]
	\centering
	\begin{forest}
		for tree={
    	child anchor=west,
    	parent anchor=east,
    	grow=east,
    	draw,
    	anchor=west,
    	edge path={
     		\noexpand\path[\forestoption{edge}]
       		(.child anchor) -| +(-5pt,0) -- +(-5pt,0) |-
       		(!u.parent anchor)\forestoption{edge label};
    		},
  	}
  	[Břetislav I.
    	[Vrastislav II.
		 		[Břetislav II.
						[Břetislav \textbf{D\footnotemark}
						]
						[Bořivoj II. (nevlastní syn)
						]
					]
				]
			[Kondrád I. Brněnský
				[Oldřich Brněnský \textbf{D}
					[Vrastislav \textbf{D}
					]
					[Vladislav
					]
				]
				[Litold Znojemský \textbf{D}
					[Kondrád II. Znojemksý \textbf{D}
					]
				]
			]
		]
		\end{forest}
		\caption[Rodokmen panovníků, vlastní tvorba]{
		\centering
			Rodokmen panovníků figurujících v rané historii kláštera, písmeno \textbf{D} značí knížata, která darovala území klášteru
		}
\end{figure}
\footnotetext{Dle opisu Kroniky Kosmovy, daruje vsi Břetislav, nejspíše se jedná o tohoto}

Bohužel je jen málo pramenů z této doby, jako jediný zdroj o rané historii kláštera máme Třebíčský (Brněnský) opis Kosmovy Kroniky\footnote{c. d. \bibentry{Uhlir1978}: 18}.
Do něho byla klášterním mnichem připsaná k roku 1115 zmínka o založení kláštera.
Mnich také uvádí jaké vsi byly postupně darovány klášteru.
Údaje jsou však sporné, a pravděpodobně vycházel z pozdějších zdrojů, neboť počet darovaných vsí je dle vpisku sedmdesát tři, což značně převyšuje i pozdější čísla.
Ve vpisku jsou uvedeny i vsi, u kterých je pozdější připojení písemně doloženo.
Existovala také klášterní kronika z roku 1201, jejímž autorem byl opat Tiburcius, ta se však ztratila.

\section{Tržní osady okolo kláštera}

Krátce po založení kláštera se začaly v jeho okolí vytvářet tržní osady.
Z nich se později zformovalo samotné město Třebíč.

\subsection{Podklášteří}

Vhodná poloha kláštera, v jeho blízkosti se nacházela řada cest, a také existence hradeb, které mohly ochránit poddané v době nepokojů, přispěly na přelomu 12. a 13. století k vzniku tržní osady.
Ta se samovolně začínala formovat přímo pod klášterem na levém břehu řeky, klášterní hradby nebyly totiž dost rozsáhlé, aby celou osadu pojaly.
Nově vzniklá osada Podklášteří ihned začala přitahovat kupce, ale i židy z širokého okolí, což vedlo k jejímu rychlému růstu.
Za nedlouho tak začala být omezována svou polohou -- skály bránily v jejím růstu a ohrožovaly ji záplavy řeky Jihlavy.
Podklášteří postupně ztrácelo na svém významu jakožto tržní osada.
Mnozí kupci proto odešli, zůstali zde však židé, kteří položily základ židovskému městu\footnote{c. d. \bibentry{Uhlir1978}: 22}.

\subsection{Vznik Starého města}
%% přepsat
V 1. třetině 12. století vznikla na příkaz klášterní vrchnosti nová tržní osada, dnešní Staré město.
Ta vznikla na pravém břehu řeky Jihlavy, kde bylo více prostoru pro rozvoj, než v již zastaveném Podklášteří.
Osada ihned začala nabývat na významu a tvořila se okolo ní aglomerace dalších osad, můžeme zmínit ves na Polánce, či vsi Kosovice, Arklebice a Horku, jenž opat Lukáš roku 1225 vyměnil se znojemskou šlechtičnou za ves Oslavany, které se nacházely dál od kláštera.
Stejně jako Podklášteří byla i nová osada omezena svým umístěním, což s postupem času opět začalo brzdit vývoj na území.\footnote{c. d. \bibentry{Uhlir1978}: 26}
Proto ještě v 1. polovině 13. století bylo založeno dnešní Nové město.


\section{Založení města Třebíč}
Město Třebíč bylo podobně jako Staré město založeno z vůle opata, jenž si vyžádal povolení u krále Přemysla Otakara II.
Z toho důvodu se také o městu Třebíč hovoříme v této době jakožto o poddanském městu náležící církevní vrchnosti.
Opat dále musel vést vyšší soudy, či rozhodoval o manství\footnote{nápravníci, lidé disponující svobodným lénem}.
Opět se jednalo o strategický krok, města totiž disponovala různými privilegii, jako právo trhu či právo hradební, právem vářečným již disponovala Staržečska\footnote{Jiné označení Starého města, dnes čtvrť v Třebíči}.
První písemná zmínka o městu Třebíč pochází z roku 1277, v ní opat Martin oceňuje lokátora Heřmana\footnote{c. d. \bibentry{Uhlir1978}: 27}, který město Třebíč založil.

\subsubsection{Přestavba kláštera}

Během 13. století dochází k rozsáhle přestavbě kláštera.
Byly vztyčeny definitivní klášterní budovy a roku 1260 byla dokončena stavba klášterního chrámu.
Společně s přestavbou budov bylo přestavěn a vylepšen i hradební systém okolo kláštera.
Bohužel během Bitvy o Třebíč\footnote{12. květen — 9. červen 1468, součást česko-uherských válek (Druhá válka husitská)} byla většina staveb zničena či poškozena.
Postupem času vznikaly i další budovy patřící klášteru rozeseté po celém klášterem spravovaném území.

\chapter{Pozdější vývoj města Třebíče}

V následující kapitole se zaměříme na pozdější vývoj města Třebíče.
A to od bitvy na Moravském poli až po konečné vítězství českých stran nad německými během národního obrození.
Třebíč se poté již vyvíjí relativně stabilně stejně jako ostatní moravská či česká města.
Dění obou válek se jí až na lidské (vyhoštění židů) a finanční ztráty příliš netýká.
Stejně tak vzrůst komunismu a jeho následný pád ji postihne stejně jako jiná česká města.

\section{Bitva na moravském poli až nástup Pernštejnů}

Klid v kraji neměl dlouhého trvání.
Kvůli bitvě na Moravském poli (26.srpen 1278) vzrostla moc šlechty, která pořádala loupeživé výpravy s cílem zmocnit se královského majetku, ohrožena byla i samotná Třebíč.
Situaci zachránil až nový král Jan Lucemburský, který sám Třebíč navštívil.
Město poté dále prosperovalo a jeho význam pomalu zastínil i samotný klášter.
Výrazně pozdějšímu vývoji pomohlo privilegium markraběte Karla (Karel IV), díky kterému město získalo pevný právní řád.\footnote{c. d. \bibentry{Uhlir1978}: 38}

Do dalších problémů se dostává město po smrti markraběte Jana Jindřicha \CrossOpenShadow 1378, té následovaly konflikty mezi Joštem a Prokopem, kteří vzájemně soupeřili o kontrolu nad Moravou.
Tyto spory vyústily až v domácí válku, která koncem 14. století vrcholí těžkou hospodářskou krizí.
Situaci uklidnil až král Václav IV. roku 1413.
Doba klidu však neměla dlouhého trvání, protože již o dva roky později se začal projevovat odpor proti církvi jako ohlas na kázání mistra Jana Husa. \footnote{c. d. \bibentry{Uhlir1978}: 28--43}

O samotném vývoji husitství v Třebíči nemáme žádné prameny, je však známo, že Třebíč udržovala vztahy se moravskými šlechtici, jež byli myšlenkám Jana Husa nakloněni.
Během Žižkova tažení na Moravu se třebíčský klášter dostal pod kontrolu husitů, z kláštera poté podnikaly výboje dál do okolí.
Konečně se situace zklidnila až v roce 1435 díky bitvě u Lipan, k které došlo v předchozím roce.
Během hustiských válek došlo k výraznému hospodářskému úpadku, ze kterého se klášter dostával celá desetiletí.
Třebíč byla po válkách věrný králi Jiřímu z Poděbrad. \footnote{c. d. \bibentry{Uhlir1978}: 49--54}

Již 12. května 1468 se Třebíč dostala do obležení, a o dva dny později byla Matyášovými (uherský král Matyáš Korvín) vojsky dobyta.
Velká část obyvatel (okolo čtyř tisíc lidé) se však uchýlila do opevněného kláštera.
6. června se česká vojska pokusila klášter zachránit, ale bez úspěchu, 15. června se posádka v klášteře definitivně vzdala.
Po celých sedm let bylo město opuštěné. \footnote{c. d. \bibentry{Uhlir1978}: 54--46}

\section{Vláda Pernštejnů}

S obnovou města se začalo teprve v roce 1474 přímo na troskách původní Třebíče.
Celková restaurace města a klášterního velkostatku však nebyla možná kvůli obrovským dluhům, kterým klášter čelil.
Nejvíce obnově napomohl Vilém z Pernštejna\footnote{bohatý rod, ve službách Jagellovců, třebíčský statek byl pod jeho kontrolou}, ten také v kraji zavedl rybnikářství a započal opravy samotného kláštera.
Roku 1507 Vilém rozdělil své území mezi syny. V závěti také apeloval na nutnost oprav různých klášterů v okolí o třebíčském nepadla ani zmínka.
Přítomnost benediktýnů mu bránila v převedení Třebíče do svého dědičného vlastnictví.
Dále se Třebíče ujímá Jan IV. z Perštejna, který pokračuje v otcově politice.
Na krátkou však dobu město přesouvá pod nadvládu Jana Jetřicha, který jako první světský správce razantně zasahuje do vedení města.
Jan z Perštejna přiděluje městu další práva, z nichž je nejdůležitější právo mílové, které značně zlepší ekonomické postavení města.
Obnovuje také na žádost městské rady dekret Jana Jetřicha Černohorského, který omezoval konkurenceschopnost židovské populace ve městě.
Po Janu (\CrossOpenShadow 1548) se vlády ujímá syn Vratislav.
Ten již Třebíč převádí do dědičného vlastnictví. \footnote{c. d. \bibentry{Uhlir1978}: 61--71}

\section{Vláda Osovců}

Mezi polovinou 16. století a rokem 1613 byl Třebíč državou Osovců, poté co Burian Osovský odkoupil třebíčské panství.
Vzhledem k ceně třebíčského panství se však Burian zadlužil, to se snažil vykompenzovat produkcí piva.
Omezoval také různá práva ve městě jako je -- právo na pronájem, ale i právo mílové.
Město proto podalo stížnost na zemský soud, ta však neuspěla, Třebíč se tedy odvolala k císaři.
Ferdinant I. poté 7. března 1560 obnovil platnost všech privilégií, zároveň napomenul Buriana, ten již dále nepodnikal tak razantní kroky proti městu. \footnote{c. d. \bibentry{Uhlir1978}: 77--78}

Roku 1568 se vlády nad městem ujal Smil Osovský z Doubravice\footnote{Jeden z důležitých pramenů doby, své paměti si zapisoval do Codexu Dubraviciana}.
Smil musel čelit dluhům svého otce, ty částečně vyřešil sňatky, ale hlavně díky razantnímu zlepšení ekonomické situace ve městě.
Zlepšení však bylo na úkor poddaných, a to v podobě vyšších rent, či delších robotních povinností.
Snažil se ale i postarat se o sirotky, předcházet požárům, zabezpečit učitele, podporovat také městský špitál, či dávat mírnější tresty -- například za pytlačení nebyla vydloubnuta oči.
Roku 1573 dále vydal Zřízení selské, které nabádalo k poslušnosti vůči úředníkům a zakazovalo sběr dříví a pastvu v panských lesích.
O devět let později získalo město právo pečetit červeným voskem a hned v příštím roce bylo vydáno městské zřízení, na jehož základě bylo město zpravováno pánovými instrukcemi.

Třebíč byla počátkem 17. století opět prosperujícím městem s rozvinutou řemeslnou výrobou a jejíž správce již nebyl zadlužen.
Během správy Smila Osovského také došlo k přestavbě kláštera na zámek.\footnote{c. d. \bibentry{Uhlir1978}: 78--96}

Po úmrtí Smila se vlády ujímá vdova Kateřina z Valdštejna, ta se sice provdala za Karla staršího ze Žerotína, ale Třebíč si nechala pod svojí kontrolou.
K městu se chovala mírně a udržovala dobré vztahy s měšťany. \footnote{c. d. \bibentry{Uhlir1978}: 101--102}

% purktretn9 knihy
\subsection{Židovské osídlení // přesunout do sólo kapitoly, není zkontrolováno}

V urbářích\footnote{soupis pro evidenci platů a dávek} se dočteme i o stavu židovské populaci ve městě.
Urbář\footnote{soupis majetku a povinností měšťanů, který tvoří šlechta}<++> z roku 1573, tedy počátku Smilovy vlády, hovoří o dvou Židech v Stařečce a šesti v Podkláštří.
Urbář z roku 1629 o jedenácti židech v podklášteří.
Během 16 století ještě židé sousedily v podklášteří s běžnými měšťany.
K růstu docházel až během konce 16. století a hlavně začátku 17 století, kdy začali skupovat domy měšťanů v podklášteří.
Samotná ekonomická situace židovské populace byla srovnatelná se zbytkem měšťanů, najdeme i zmínky i zadluženosti některých židů.
V urbářích se neobjevují ani jako věřitelé, až na jednu zmínku, či nejsou zaznamenány žádné ilegalní obchody.
Můžeme se domýšlet že již během 16. století existovala v Třebíči synagoga, protože počátkem 17. století vzniká synagoga s přízviskem nová. \footnote{c. d. \bibentry{Uhlir1978}: 89}

\section{Třicetiletá válka}

I přes nátlak ze strany krajanů zůstal Žerotín nestranný a defenestraci Čechů neschvaloval jako adekvátní.
Žerotín nepodpořil ani krále Fridricha, zimního krále, v jeho úspěch totiž nevěřil, naopak samotná Třebíč zimnímu králi nabídla nocleh při jeho cestě do Brna a předala dary.
O dalším předbělohorském vývoji nejsou žádné prameny.
Karel starší ze Žerotína také slíbil ochranu města a 9. 12. 1620 sám císař Ferdinant II. zaručil, že císařská armáda město nepoškodí a bude ho chránit.
Mezi 14.-15. prosincem 1620 tak městem císařská armáda jen prošla a zanechala zde ochrannou posádku.
Roku 1621 proto Třebíč dokázala vyhovět prosbám o pomoc měst Jihlavy a Znojma, která pod ochranou císaře nebyla. \footnote{c. d. \bibentry{Uhlir1978}: 103--106}

Počátkem ledna 1622 do města přijel Marradasův jezdecký pluk císařské armády o síle pěti set jezdců, který představoval pro Třebíč značnou zátěž.
Pluk se navíc nijak nehrnul do bojů a v městě pobyl až do října.
Potenciální nebezpečí pro město představoval Gábor Bethlen poté co začal podnikat nájezdy na moravské území a pustošit místní města, Třebíč proto požádala o pomoc.
Gábor však začal mírová jednání předtím, než byla Třebíč přímo ohrožena.
Mezitím však dorazily dva regimenty slezského vojska o síle tří tisíc vojáky, ti v Třebíči tábořily dvacet tři dní.
Roku 1626 došlo k dalšímu uherskému výboji, což vyvrcholilo opět v pobyt císařských vojsk v Třebíči. \footnote{c. d. \bibentry{Uhlir1978}: 106--107}

Další ránu městu zasadilo Obnovené zřízení zemské.
Po něm již náboženskou svobodu, kterou Třebíč dříve disponovala,  nemohla šlechta nijak garantovat\footnote{Na Moravě byla náboženská svoboda dlouho standardem, proto se jí netýkal ani Rudolfův majestát.}.
Brzy po vydání emigračního patentu (9. březen 1928) se proto Žerotín rozhodl pro emigraci, konkrétně do Vratislavi\footnote{Činní tak v rámci pomoci nově persekuovaným, zvláště Jednotě bratrské}.
Z Vratislavi mohl stále spravovat území, které si ponechal, a Kateřina stále byla schopna dohlížet na Třebíčí.
Novým držitelem Třebíče se na jaře roku 1928 stává Kateřin bratr Adam z Valdštejna, samotná Kateřina roku 1638 zemřela a o dva roky dříve zemřel Karel starší ze Žerotína. %
Jejich těla jsou pochována v Třebíči. % 108 -110

Adam však nejevil o správu města velký zájem a přenechal ji synu Rudolfovi z Valdštejna.
Ten již neměl takové postavení politické jako jeho otec a ani příliš nezasahoval do náboženských poměrů ve městě.
Otázky náboženství nepovažoval za příliš důležité a spíše se zajímal o otázky ekonomické, které řešil vysokými poplatky. \footnote{c. d. \bibentry{Uhlir1978}: 109--112}

Vývoj třicetileté války během přechodu z dánské fáze do fáze švédské příliš Třebíč neovlivnil.
Až tažení na Moravu Linharta Torstensona\footnote{Švédský polní maršál, velmi schopný a obratný vojevůdce.} Třebíč postihlo
Roku 1644 se musela podílet na vydržování a vystrojení vojsk, která bránila Olomouc.
V březnu 1645 však do města přišla švédská jednotka, ta si vynutila 6500 zl. výpalného a obnovila v nekatolické bohoslužby, musela také platit poplatek švédské posádce v Jihlavě. \footnote{c. d. \bibentry{Uhlir1978}: 112--113}

\section{Obobí rekatolizace}

Moravská šlechta pod vedením Jana Rottala\footnote{Moravský šlechtic z rodu Rottalů, držitel řádu zlatého rouna a jedna z významných osobností habsburské monarchie.} začala vydávat jedno nařízení proti nekatolíkům za druhým, stejně se choval i císař se svými pateny z roku 1651.
V téže roce byla i sestavena komise s cílem provádět inspekci církevního majetku na Moravě.
Stejně jako pro další moravské města tak i pro Třebíč znamenala druhá polovina 17. století období rekatolizace až do 18. století však tajné nekatolictví v městě vydrželo.
Došlo také k obnovení cechů, které postupně eliminovaly konkurenci a začali brzdit vývoj v městě. \footnote{c. d. \bibentry{Uhlir1978}: 125--126}

Samotné město se dostalo pod správu Adama Františka, jeho ze začátku zastupovala Zdislava ze Sezimova Ústí.
O Zdislavině působení nemáme příliš písemností, víme však, že odpustila dluh kupcově rodině a vyhověla stížnosti Židů, aby nebyli vystavováni násilí nestihnou--li robotu.
30. června 1655 se vlády ujal syn Adam, ten se choval podobně jako jeho otec a otázce náboženství tak nepřikládal velkou hodnotu.
Již však nebyl schopen odolávat náporu z venčí a zvláště žalobě polenského děkana Kryštofa Kazimíra Burešovského, ta byla podložena řadou důkazů obsahovala však i polopravdy, či naprosté lži, jako bylo obvinění z existence 21 sekt v Třebíči.
Právě tato žaloba vyvolala na úřadech velkou reakci, což vyvrcholilo vytvořením \uv{inkviziční komice} zorganizované 2. prosince 1656 novoříšským proboštem Engelbertem a moravskobudějovickým Rudolfem Jindřichem ze Schaumburka.
Ta působila ve městě od 3. do 9. února 1657, za přítomnosti rychtáře, faráře a i samotného žalobce. \footnote{c. d. \bibentry{Uhlir1978}: 126--128}

Komise například zjistila, že v Podklášteří žilo 13 nekatolických rodin ku 7 katolickým a podobně na tom byl zbytek Třebíče.
Nejvíce dominovali utrakvisté, vyskytovali se však i luteráni.
Závěr komise je byl relativně jasný -- Třebíč je útočištěm nekatolíků, zvláště zběhlých z Čech.
Komise si také stěžovala na chování židů a obvinila také město z útlaku katolíků, proti tomu se však městská rada ihned ohradila.
Děkan Burešovský a farář Slovatius navíc přidali další stížnosti vůči Třebíči -- závěrečná zpráva tak pro Třebíč nebyla nikterak dobrá.  \footnote{c. d. \bibentry{Uhlir1978}: 128--129}

Již 8. května 1657 přichází do města protireformační komise vedená hejtmanem Tannazollem.
Její prvotní snahy však byly neúčinné, celkově od listopadu 1658 do července 1659 se podařilo obrátit na \uv{pravou} víru jen 109 lidí, obecně lidé neměli přílišnou motivaci, pokuty pro nekatolíky nikdo nevymáhal.
Přelomem 50. a 60. let, i přes nezdárné úspěchy komise, došlo k nová vlně emigrace a 20\% šlechty opustilo Třebíč. \footnote{c. d. \bibentry{Uhlir1978}: 130--131}

Počátkem 60. let začalo také hrozit nebezpečí se strany Turků, kteří pořádali vpády na Moravu.
Roku 1664 však porážkou Turků u svatogotthardského průsmyku ohrožení pominulo. \footnote{c. d. \bibentry{Uhlir1978}: 132--133}

Po Adamově smrti (\CrossOpenShadow 1666) připadá město Františkovi Augustinovi.
Za jeho vlády se dostává rekatolizace do nové fáze, František zároveň usiluje o zřízení kapucínského kláštera, jeho žádosti však nebylo vyhověno.
Jeho správa neměla dlouhého trvání a brzy se na jeho místo dostává bratr Karel Ferdinand.
Ten se, na rozdíl od jeho předchůdců, zapsal jakožto dobrotivý pán.
Během jeho vlády do města přicházejí i zástupci kapucínů a rychle se začleňují do městské společnosti. \footnote{c. d. \bibentry{Uhlir1978}: 133--135}

\section{Spory mezi měšťany a vrchností}

Za vlády Josefa I. docházelo k řadě sporů mezi měšťany a vrchností.
Ty zakončil až reskript císaře z 15. prosince 1708.
V něm císař ponechal Třebíči soudní pravomoc, zároveň však apeloval, aby se Valdštejn na jurisdikci podílel a zaručil také měšťanům právo na svobodný trh. \footnote{c. d. \bibentry{Janak1981}: 28--29}

Mezi roky 1724 až 1726 však došlo k dalším sporům mezi městem a vrchností -- Janem Josefem z Valdštejna.
Ty vyvrcholily v narovnání roku 1726, které posunulo město dál k emancipaci měšťanů. \footnote{c. d. \bibentry{Janak1981}: 29--33}

K dalšímu sporu došlo roku 1728, kvůli termínu tělesné člověčenství (vyjádření závislosti na feudálovi) jenž v jedné listině Jan Josef použil, tento spor měšťané prohrály a opět se objevila idea, že vztah měšťana a pána je vždy \uv{nevolnický}.
Spor se dostal před tribunál ještě jednou díky kupci Janu Bernardovi Matesovi, který odmítl přijmout \uv{nevolnictví}, načež mu bylo odebráno právo obchodovat, což bylo v rozporu s právem na svobodný trh, vrchnost zde již byla nucena ustoupit.
Po smrti Jana Josefa se stal držitelem města František Heřman Arnošt z Valdštejna, ten se výrazu nevolnictví plně zřekl a nepoužily ho ani jeho následovníci. \footnote{c. d. \bibentry{Janak1981}: 33--35}
Spory se vyskytovaly i za vlády Marie Terezie.

\section{Vláda Marie Terezie}

Za vlády Marie Terezie se postupně začalo opouštět od tradiční ekonomice feudální společnosti a jít vstříc kapitalistickému fungování.
V Třebíči se vývoj odehrával podobně a na důležitosti nabyli zejména tkalci, ševci a koželuži, oproti tomu upadlo pro Třebíč dlouhou dobu charakteristické soukenictví, které dostalo do problémů z důsledku nedostupnosti vlny, což mělo za následek i soudní spor.
Ve městě také vznikla papírna.
Můžeme zmínit, že obchodem se živila většina židovských obyvatel v roce 1973 to bylo 55, v roce 1799 již 92.
Samotná průmyslová revoluce však přišla do Třebíče trochu opožděně. \footnote{c. d. \bibentry{Janak1981}: 13--29}

Třebíč se stala obětí válek o děditství rakouské, kdy 10. února 1742 do města vtrhli pruští husaři a muselo jim být zaplaceno 12 000 zl. výpalného a 6000 zl. navíc od židů.
Dále 12. února dorazila armáda krále Bedřicha Velikého o síle 12 000 mužů, kterou město muselo ubytovat a 24. února.
Devastující byly také neúrody mezi léty 1765 a 1769, dále 1771 a 1772, ty vyvrcholily až v selské povstání z roku 1775. \footnote{c. d. \bibentry{Janak1981}: 38}

1. ledna 1755 přišla Třebíč o právo hrdelní.
Tato změna souvisela s reformou, která omezovala počet institucí, jenž tento trest mohly vynést. \footnote{c. d. \bibentry{Janak1981}: 39}

\section{Vláda Josefa II.}

První Josefův zásah byl prostřednictvím nového dekretu o soudnictví z 27. prosince 1786, podle kterého mělo být soudnictví soustředěno do kriminální soudů, jako které fungovaly magistráty krajských měst, v případě Třebíče to byla Jihlava.
Také došlo k zrušení městských rad a jejich nahrazení magistráty, jejichž zastupitelé -- purkmistr a rada volilo měšťanstvo prostřednictvím měšťanského výboru, samotné kandidáty však registrovala vrchnost, což jí dávalo větší moc, než jakou mělo měšťanstvo.
Spory s vrchností však neustaly ani po vytvoření magistrátu (1. volby se konaly v březnu roku 1787), vrchnost také kromě schvalování kandidátů rozhodovala o výdajích města.
Docházelo také k silné germanizaci ve správě města, a tak veškerý spisy z této doby jsou v němčině. \footnote{c. d. \bibentry{Janak1981}: 36--37}

\section{Napoleonské války}

Dne 17. srpna 1776 byl zveřejněn patent vyzývající občany k vojenské službě, samotné město dalo na válečnou snahu značný peněžní příspěvek.
To i přesto, že na přelomu 18. a 19. století bylo obecní hospodářství ve velmi špatném stavu.
Mezi 19. a 26. listopadem 1805 prošel městem první sbor pod vedením maršála Bernadottema, toho uvítal farář Dvořecký a nabídl Bernadottemu nocleh na zámku, ten se však spokojil s domem radního.
Za nedlouho však ve městě vypukl nedostatek jídla a francouzští vojáci museli táhnou dál.
Dále se ještě musela Třebíč postarat o dvě divize, které sem přišli po bitvě u Slavkova a pobývaly v Třebíči mezi 9. až 29. prosincem.
Celkové výdaje dosáhly 300 000 zl., ztráta nikdy nebyla vykompenzována.
Situaci navíc ještě zhoršily špatné úrody v letech 1807 a 1808. \footnote{c. d. \bibentry{Janak1981}: 38--39}

Po uzavření příměří po bitvě u Znojma Třebíč protínala demarkační linie, což způsobilo městu další finanční ztráty o celkové hodnotě 100 000 zl.
Navíc do města byly svezeni raněních obou bojujících stran, to mělo za následek tyfovou epidemii. \footnote{c. d. \bibentry{Janak1981}: 39}

\section{Vývoj po napoleonských válkách}

Další devastující událostí pro Třebíč byl státní bankrot z roku 1811, který obral malé živnostníky i o to málo co jim zbylo, mezi mocnými se také rychle rozšířila korupce.
V Třebíči navíc byly osočeni židé, že o bankrotu věděli předem a začali si skupovat majetek.
Do ohně ještě přilili neúrody v letech 1815 a 1817 a vše završila morová epidemie mezi dobytkem roku 1819, po které přestala fungovat většina řemeslné výrovy ve městě. \footnote{c. d. \bibentry{Janak1981}: 39--40}

Vše mělo za následek robotní vzpouru roku 1821, ta situaci nijak nezlepšila.
Vzbouřenci navíc byly odvedeni na zámek kde je ztrestali holemi, městem poté začaly kolovat letáky vyzývající k ukončení trestů s hrozbou žhářství.
Hrozby se stali skutečností a dne 3. května shořelo 100 domů v židovské obci a dalších 50 v Podklášteří. \footnote{c. d. \bibentry{Janak1981}: 30}

V červnu roku 1822 město opět zachvátil požár, který měl více devastující účinky než bitva o Třebíč.
Během červnových požárů shořelo přes 290 domů a to včetně kostela, radnice, fary a jedné školy.
Třebíč se poté opět dostala do finančních problému, které vyřešil až syndik Gottsmann.

Třebíči však nebylo dovoleno dlouhému oddychu, v roce 1830 byla Třebíč zasažena povodní, v roce 1832 a opět v roce 1836 vypukla epidemie cholery a v roce 1847 město opět zasáhl požár.

\section{Židovské město // přesunout do sólo kapitoly, není zkontrolováno ani dokončeno}

Na přelomu 17. a 18. století byly odkoupeny zbylé křesťanské domky v Podklášteřží a začal se pro něj používat termín židovské město, který předtím popisoval jen jeho část.
Podkálšteří fungovalo jako samostatná obec a nespadalo ani do stejné jurisdikce jako Třebíč.
V čele samotného židovského města stála rada v jejímž čele byl židovský rychtář, jenž byl volen občany města.

// poznámka: strany 44 - 45, jen židovské město, více informací, zahrnu do kapitoly o samotném židovském městě.


// poznámka: strany 49 - 64, jen řemeslná výroba a soudní spory ohledně ní, něco málo o zavádění strojů, ale vcelku irelevantní k mé práci, nehledě na to že od Marie Terezie nebyla žádná změna co se týče řemeslných oborů

\section{Národní obrození}

Snaha o národní obrození se v Třebíči začala objevovat ve 20. letech 19. století.
Většina obyvatel Třebíče byla česká, a případní cizinci (manželky) byly do českého prostředí rychle asimilováni, správu nad městem však držely němečtí úředníci a tak tempo obrození nebylo nijak valné.
První náznak změny představoval rodák Josef Chmela, spisovatel a obdivovatel Josefa Jungmana, ten v Třebíči pobýval o svých prázdninách kdy v hospodě zorganizoval malý debatní kroužek, jinak ale působil v Praze.
Jeho krátké návštěvy však neměly velký vliv, rozhodující osobností se stal až doktor Jan Miloslav Haněl.
Ten do Třebíče přišel jako mladý lékař roku 1835 z Prahy, a o devět let později založil čtenářský spolek, angažoval se také v divadelní činnosti, kterou pořádal spolu s českými studenty. \footnote{c. d. \bibentry{Janak1981}: 33--35}\footnote{c. d. \bibentry{Janak1981}: 67--69}

Samotné události roku 1848 zastihli vedení Třebíče nepřipravené, situaci stabilizovalo až vytvoření národní gardy, to ale také vedlo k rozdělení měšťanstva na stranu německou, v jejímž čele stál Ignác Glas, a na stranu českou s dr. J. M. Hanělem v čele.
Zatím však národní garda pracovala společně s Němci. \footnote{c. d. \bibentry{Janak1981}: 69--70}

1. českou manifestací se stalo svěcení praporu národní gardy uspořádané dne 28. května 1849, při něm se zpívaly české písně a provolávala různá hesla.
Garda však neměla dlouhého trvání a již koncem roku 1849 musela z nařízení ukončit svoji činnost. \footnote{c. d. \bibentry{Janak1981}: 70--72}

20. října 1860 byl vydán Říjnový diplom, ve kterém byla zaslíbena konstituční vláda, načež 21. února 1861 byla vydána samotná ústava.
Obě události měšťané přijali s nadšením, a záhy se začala angažovat v Třebíči česká národní strana a vytvořila spolek Měšťanská beseda s dr. J. M. Hanělem v čele.
Jednalo se o jeden z prvních podobných národních spolků na území Moravy.
Německá Třebíč jako odpověď na český spolek utvořila organizace Gesangverein, Turnverein a Schützverein (původní střelecký spolek). \footnote{c. d. \bibentry{Janak1981}: 76}

Období 40. let se také neslo v duchu sociální nejistoty drobných výrobců, ta vyvrcholila vystoupeními namířenými hlavně proti židovské čtvrti.
Během let 48-49 došlo k dočasnému zrovnoprávnění, což vedlo i k přesun zámožných židů z židovského města na náměstí.
To však mělo za následek další bouře proti židovskému obyvatelstvu, k těm došlo mezi 1. dubnem a 7. květem 1850.
K dalším nepokojům došlo v roce 1860, společně úpadkem pro Třebíč charakteristického soukenictví.
Město také trpělo akutním nedostatkem peněz, byly proto vydány dočasné bankovky, ty se však velmi lehce padělaly.
Celá řada měšťanů se dostala do finančních problémů a byla nucena prodat své domy.
To mělo za následek další proti židovské nepokoje (1863 a 1866). \footnote{c. d. \bibentry{Janak1981}: 77--78}

\section{Prusko-rakouské války}

Během Prusko-rakouské války došlo k poslednímu vzrůstu soukenictví, ekonomické situaci měšťanstva to však nijak nepomohlo, zakázky z armády totiž doprovázely i rekvizice, které město přišli až na 250 000 zl.
Třebíč se navíc dočkala opět nepřátelské invaze, a to 13. července, kdy do města přitáhla pruská pěchota, dělostřelectvo a jezdectvo, o síle 35 000 mužů.
Tato posádka v Třebíči pobyla do 16. prosince.
Dále město hostilo záložní sbory a zdravotní kolony.
Okupováno bylo až do uzavření příměří, poslední jednotky odešli až 8. září.
Pruské jednotky do města navíc zavlekli choleru. \footnote{c. d. \bibentry{Janak1981}: 78}

\section{Další vývoj}

20. srpna 1867 Třebíč dostala svého českého starostu -- Ignáce Přerovského, zároveň radními příslušníky se stali členi Měšťanské besedy jako byl J. M. Haněl.
Vznikla také Občanská záložna, která měla fungovat jakožto finanční opora českého národního hnutí.
K založení Sokola však v Třebíči nedošlo. \footnote{c. d. \bibentry{Janak1981}: 79--83}

Při dalších volbách roku 1870 Německá strana vůbec nekandidovala a do obecního výboru tak byly zvoleni jen čeští funkcionáři.
Situace však neměla dlouhého trvání, již v roce 1871 se moravským místodržitelem stal baron Weber, který opět začal potlačovat národní snahu a rozpustil i moravský zemský sněm.
Němců se zastala i většina židovského města a v dalších volbách byl jako starosta zvolen představitel německých liberálů -- dr. Jan Bažant.
Krátce po jeho zvolení byla definitivně ukončena činnost záložny, a o rok později bylo zrušeno i obecní zastupitelstvo, na základě obvinění z mrhání penězi. \footnote{c. d. \bibentry{Janak1981}: 82--83}

Správu nad městem dočasně poté převzala komise a ta se postarala o uspořádání voleb, ty byly zmanipulované v neprospěch České národní strany.
7. prosince 1873 bylo ustanoveno nové obecní zastupitelstvo a Česká národní strana byla v menšině, do samotné městské rady se nedostal jediný český zástupce.
Příštích voleb se už čeští zástupci nezúčastnily vůbec a ve volbách v roce 1879 bylo vedení města opět z většiny německé. \footnote{c. d. \bibentry{Janak1981}: 83--90}

// poznámka: byly změny v židovském městě a samotní židé hodně ovlivňovaly politiku Třebíče, budu se věnovat v samostatné kapitole o městě

Ke změně došlo až u voleb z roku 1879, kdy česká strana získala naprostou většinu hlasů a Třebíč opět dostal českého starostu.
Jednalo se o první město velké město na Moravě, kterému se něco podobného podařilo.
Třebíč však byla opět ve finančních problémech, německá vláda zanechala dluh 107 727 zl. (celkový dluh činil 123 455 zl.).
Byly zvýšeny přirážky o celých 50\%, a to k nelibosti pracujících, stát zároveň začal dávat větší důraz na místní samosprávu, věc na kterou město vůbec nemělo prostředky. \footnote{c. d. \bibentry{Janak1981}: 93--94}

Konec 19. století a začátek 20. byl zároveň doprovázen s tvorbou různých spolků a hnutí, mezi nejdůležitější patřilo dělnické, to v roce 1898 vydalo rezoluci požadující rovné volební právo, ochranu chudých, zřízení burzy práce.
Městský výbor sice požadavkům nedával velkou váhu, spolky však měli i tak dopad dění ve Třebíči. \footnote{c. d. \bibentry{Janak1981}: 97--115}

// poznámka: další kapitola se zaobírá průmyslem, informace o židech, bude zpracováno v sólo kapitole. Další nese název Třebíčtí socialisté v čele dělnického hnutí západní Moravy a je čistá propaganda (více než zbytek knihy), další se zabývají kulturou a školství, bez jediné zmínky o židech

\chapter{Dějiny židovského města v Třebíči}

\chapter{Aktuální stav židovského města v Třebíči}

\chapter*{Závěr}
\addcontentsline{toc}{chapter}{Závěr}

Cíl práce se i přes problémy podařilo splnit.
Boužel se mi díky epidemii CoVid--19 nepodařilo město navštívit a získat tak mé osobní fotografie.
Ty však tvoří jen malou část práce a během nočního přejezdu jsem ve městě zastavil a alespoň prošel.

Jediný problém ke zpracování představovalo období dějin města od revolučního roku 1848 po současnost.
Bohužel v tomto časovém rozsahu nejsou ani publikace ohledně samotného Třebíče.
Byla sice plánována kniha Dějiny města III., k jejímu vytvoření snad nikdy nedošlo, či vzniklo jen pár výtisků.
Obecně nedostatek středoškolskému studentu přístupných pramenů komplikuje popsat dějiny židovské komunity v Třebíči podrobně.

I tak je výsledkem této práce obstojný souhrnný pohled na dějiny židovské obce ve Třebíči.
Sám jsem díky této práci získal hluboký náhled do historie nejen samotného Židovského města, ale i celého Třebíče a Moravy.
Pevně doufám, že se tohoto tématu ujme schopný historik a celé dějiny zmapuje podrobně od začátku do konce.


\include{literatura}

\listoffigures
\openright
\end{document}
