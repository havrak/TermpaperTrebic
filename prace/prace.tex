%%% Hlavní soubor. Zde se definují základní parametry a odkazuje se na ostatní části. %%%

%% Verze pro jednostranný tisk:
% Okraje: levý 40mm, pravý 25mm, horní a dolní 25mm
% (ale pozor, LaTeX si sám přidává 1in)
\documentclass[a4paper,oneside,12p]{report}
%\documentclass[a4paper,oneside,12p,twoside]{report}
\setlength\textwidth{155mm}
\setlength\textheight{237mm}
%\setlength\oddsidemargin{00mm}
\setlength\oddsidemargin{05mm}
\setlength\evensidemargin{05mm}
\let\openright=\clearpage


%% Vytváříme PDF/A-2u
\usepackage[a-2u]{pdfx}

%% Přepneme na českou sazbu a fonty Latin Modern
\usepackage[czech]{babel}
\usepackage{lmodern}
\usepackage[T1]{fontenc}
\usepackage{textcomp}

%% Použité kódování znaků: obvykle latin2, cp1250 nebo utf8:
\usepackage[utf8]{inputenc}

%%% Další užitečné balíčky (jsou součástí běžných distribucí LaTeXu)
\usepackage{amsmath}        % rozšíření pro sazbu matematiky
\usepackage{amsfonts}       % matematické fonty
\usepackage{amsthm}         % sazba vět, definic apod.
\usepackage{bbding}         % balíček s nejrůznějšími symboly
			   										% (čtverečky, hvězdičky, tužtičky, nůžtičky, ...)
\usepackage{bm}             % tučné symboly (příkaz \bm)
\usepackage{graphicx}       % vkládání obrázků
\usepackage{fancyhdr}				% možnost slylizovat záhlaví
\usepackage{fancyvrb}       % vylepšené prostředí pro strojové písmo
\usepackage{indentfirst}    % zavede odsazení 1. odstavce kapitoly
\usepackage[nottoc]{tocbibind} % zajistí přidání seznamu literatury,
\usepackage{icomma}         % inteligetní čárka v matematickém módu
\usepackage{dcolumn}        % lepší zarovnání sloupců v tabulkách
\usepackage{booktabs}       % lepší vodorovné linky v tabulkách
\usepackage{paralist}       % lepší enumerate a itemize
\usepackage{caption}				%	popisky
\usepackage{dirtree}				% strom souborů
\usepackage[bottom]{footmisc}  % poznámky pod čarou vespod
\usepackage{bibentry}
\nobibliography*

\usepackage{color}
\usepackage{forest}					% na rodokmeny
\usepackage{natbib}
%\setbibentrystyle{author}

\definecolor{pblue}{rgb}{0.13,0.13,1}
\definecolor{pgreen}{rgb}{0,0.5,0}
\definecolor{pred}{rgb}{0.9,0,0}
\definecolor{pgrey}{rgb}{0.46,0.45,0.48}

%%% Údaje o práci

\def\NazevSkoly{Gymnázium, Praha 6, Arabská 14}
% Název oboru včetně počátečního 'Obor'.
\def\NazevOboru{Dějepis}

% Název práce v jazyce práce (přesně podle zadání)
\def\NazevPrace{Židovké město v Třebíči}

% Název práce v angličtině
\def\NazevPraceEN{Jewish town in Třebíč}

% Název práce v němčině
\def\NazevPraceDE{Jüdische Stadt in Třebíč}

% Jméno auto
\def\AutorPrace{Havránek Kryštof 2.E}

% Rok odevzdání
\def\RokOdevzdani{2020}
% Měsíc odevzdání
\def\MesicOdevzdani{Květen}

% Vedoucí práce: Jméno a příjmení s~tituly
\def\Vedouci{PhDr. Lenka Dvořáková }

% Nepovinné poděkování (vedoucímu práce, konzultantovi, tomu, kdo
% zapůjčil software, literaturu apod.)
\def\Podekovani{%
\textbf{Poděkování}

TODO

PhDr. Lenka Dvořáková

Milan Krčmář (zjisit si titul!)
}

% Abstrakt (doporučený rozsah cca 80-200 slov; nejedná se o zadání práce)
\def\Abstrakt{%
TODO
}
\def\AbstraktEN{%
TODO
}
\def\AbstraktDE{%
TODO
}

% 3 až 5 klíčových slov (doporučeno), každé uzavřeno ve složených závorkách
\def\KlicovaSlova{%
{klíčová} {slova}
}
\def\KlicovaSlovaEN{%
{key} {words}
}


%% Balíček hyperref, kterým jdou vyrábět klikací odkazy v PDF,
%% ale hlavně ho používáme k uložení metadat do PDF (včetně obsahu).
%% Většinu nastavítek přednastaví balíček pdfx.
\hypersetup{unicode}
\hypersetup{breaklinks=true}

%% Definice různých užitečných maker (viz popis uvnitř souboru)
\include{makra}

%% Titulní strana a různé povinné informační strany
\fancypagestyle{plain}{
\fancyhead[C]{}
\fancyhead[L]{Ročníková práce - Gymnázium, Praha 6, Arabská 14}
\fancyhead[R]{\textbf{Židovské město v Třebíči}}
\fancyfoot[L]{Vypracoval: Havránek Kryštof 2.E (Programování)}
\fancyfoot[C]{}
\fancyfoot[R]{\thepage}
\renewcommand{\headrulewidth}{0.4pt}
\renewcommand{\footrulewidth}{0.4pt}
}



\begin{document}

\include{titulka}

\tableofcontents
\newpage


\chapter*{Úvod}
\addcontentsline{toc}{chapter}{Úvod}

Židovské město ve Třebíči jsem si vybral pro jeho unikátnost.
Na rozdíl od většiny leželo židovské osídlení už při svém vzniku přímo v centru města.
Navíc řada budov zůstala zachována ve velmi dobrém stavu.
Zároveň stavby samotné jsou architektonicky velice zajímavé, po tvorbě ghetta se město nemohlo dál rozšiřovat a vznikla tak velmi hustá, průchody protkaná, zástavba.
Třebíč je místem, které jsem sám dvakrát navštívil a velice mě zaujalo jako malebné moravské městečko.
V neposlední řadě mě lákal koncept tématu této práce -- zmapovat staletí dějin jednoho malého místa.

Cílem práce je prozkoumat jak dějiny Židovského města, tak dějiny Třebíče.
Poté se zaměřit na unikátní stránku města -- Židé a křesťané žili v těsném soužití, denně se zde střetávaly dva odlišné světy, jak toto soužití probíhalo.
Nakonec projít historii některých vybraných staveb Židovského města jako třeba dvě synagogy, jež se v Třebíči nachází.


\pagenumbering{arabic}
\setcounter{page}{1}

\chapter{Počátky města Třebíče}

\section{Charakteristika města Třebíče}

Dnes je Třebíč nepříliš velké město v jihovýchodní části kraje Vysočina.
I přes jeho velikost, však hrálo důležitou roli v historii okolí.
O tom svědčí i fakt, že má dokonce dvě památky zapsané v seznamu UNESCO a to Třebíčskou židovskou čtvrť a baziliku svatého Prokopa.
Počátky jeho historie jsou však, spíše než s národem židovským, spjaty s benediktýnským klášterem, dnešním zámkem v Třebíči.
Je možné, že na území Třebečského lesa se nacházelo i prehistorické osídlení, o tom však nemáme důkazy.

\section{Vznik kláštera}

Dějiny kláštera se začínají psát koncem 11. století.
Jednalo se o dobu kdy na českém i moravském území docházelo k řadě konfliktů mezi jednotlivými Přemyslovci a moravskými knížeti, právě ty měly za následek založení kláštera.
Nejvíce k založení kláštera napomohlo chování Břetislava II., který jako svého následníka jmenoval svého nevlastního syna Bořivoje II., místo brněnského knížete Oldřicha\footnote{ten měl být králem dle rodové tradice Přemyslovců}.
Kvůli této a dalším neshodám uspořádal roku 1099 král Břetislav tažení na brněnský hrad, což donutilo Oldřicha a jeho bratra Litolda, jenž spravoval znojemský úděl, k útěku.
Po dvou letech vlády Bořivoje se však Oldřich a Litold opět vrátili na své úděly, využili šance kdy se po smrti Břetislava II. Bořivoj II. musel vrátit do Prahy.

Již při cestě zpět si uvědomily, že západní hranice jejich údělu jsou příliš daleko od hradů v Brně a Znojmě.
Z tohoto důvodu v roce 1101 založili v prostorách Třebečského lesa téměř na hranicích s Přibyslavickem benediktýnský klášter.
Klášter měl sloužit jako předsunuté opevnění a kolonizační centrum.
Již v roce 1101\footnote{c. d. \bibentry{Uhlir1978}} byl postaven kostelík sv. Benedikta, který o tři roky později vysvětil pražský biskup Heřman.
Dále roku 1109\footnote{c. d. \bibentry{Uhlir1978}: 20} vysvětil biskup olomouckým Jan II. samotný klášterní chrám Nanebevztí Pany Marie.
Klášter ihned po založení dostával různá územní darování, a to jak od Oldřicha a Litolda, tak i od jejich nástupců.
Samotný Litold (\CrossOpenShadow 1112) a Oldřich (\CrossOpenShadow 1113) jsou dle slov Tomáše Pěšinky\footnote{Tomáš Petřinka z Čechorodu -- biskup, dějepisec, autor díla Mars Moravicus} pochováni v klášteře.

\begin{figure}[h]
	\centering
	\begin{forest}
		for tree={
    	child anchor=west,
    	parent anchor=east,
    	grow=east,
    	draw,
    	anchor=west,
    	edge path={
     		\noexpand\path[\forestoption{edge}]
       		(.child anchor) -| +(-5pt,0) -- +(-5pt,0) |-
       		(!u.parent anchor)\forestoption{edge label};
    		},
  	}
  	[Břetislav I.
    	[Vrastislav II.
				[Břetislav II.
						[Břetislav \textbf{D\footnotemark}
						]
						[Bořivoj II. (nevlastní syn)
						]
					]
				]
			[Kondrád I. Brněnský
				[Oldřich Brněnský \textbf{D}
					[Vrastislav \textbf{D}
					]
					[Vladislav
					]
				]
				[Litold Znojemský \textbf{D}
					[Kondrád II. Znojemksý \textbf{D}
					]
				]
			]
		]
		\end{forest}
		\caption[Rodokmen panovníků, vlastní tvorba]{
		\centering
			Rodokmen panovníků figurujících v rané historii kláštera, písmeno \textbf{D} značí knížata, která darovala území klášteru
		}
\end{figure}
\footnotetext{Dle opisu Kroniky Kosmovy, daruje vsi Břetislav, nejspíše se jedná o tohoto}

Bohužel je jen málo pramenů z této doby, jako jediný zdroj o rané historii kláštera máme Třebíčský (Brněnský) opis Kosmovy Kroniky\footnote{c. d. \bibentry{Uhlir1978}: 18}.
Do něho byla klášterním mnichem připsaná zmínka k roku 1115 o založení kláštera.
Mnich také uvádí jaké vsi byly postupně darovány klášteru.
Údaje jsou však sporné a pravděpodobně vycházel z pozdějších zdrojů, neboť počet darovaných vsí je dle vpisku sedmdesát tři, což značně převyšuje i pozdější čísla.
Ve vpisku jsou i uvedeny vsi u kterých je pozdější připojení písemně doloženo.
Existovala také klášterní kronika z roku 1201, jejímž autorem byl opat Tiburcius, ta se však ztratila.

\section{Tržní osady okolo kláštera}

\subsection{Podklášteří}

Vhodná poloha kláštera (procházelo okolím kláštera řada cest) a existence hradeb přispěly na přelomu 12. a 13. století k vzniku tržní osady.
Ta se samovolně začínala formovat přímo pod klášterem na levém břehu řeky, klášterní hradby nebyly totiž dost rozsáhle aby celou osadu pojaly.
Nově vzniklá obec Podklášteří ihned začala přitahovat kupce, ale i židy z širokého okolí, což vedlo k jejímu rychlému růstu.
Za nedlouho tak začala být omezována svým umístěním, byla totiž obehnána skálami a ohrožována záplavami řeky Jihlavy.
Podklášteří proto postupně ztrácelo na svém významu jakožto tržní osada a mnozí kupci odešli, zůstaly zde však židé, kteří položily základ židovskému městu\footnote{c. d. \bibentry{Uhlir1978}: 22}.

\subsection{Vznik Starého města}
%% přepsat
V 1. třetině 12. století vznikla na příkaz klášterní vrchnosti nová tržní osada, dnešní Staré město.
Ta se tentokrát nacházela na pravém břehu řeky Jihlavy, což umožňovalo více prostoru pro rozvoj.
Ihned po vytvoření se také začala tvořit aglomerace okolo nové osady, můžeme zmínit ves na Polánce, či vsi Kosovice, Arklebice a Horku, jenž opat Lukáš roku 1225 vyměnil se znojemskou šlechtičnou za ves Oslavany, které se nacházely dál od kláštera.
Stejně jako Podklášteří byla i nová osada omezena svým umístěním, což s postupem času opět začalo brzdit vývoj na území.
Proto ještě v 1. polovině 13. století bylo založeno Nové město.


\section{Založení města Třebíč}
Město Třebíč bylo podobně jako Staré město založeno z vůle opata, jenž si vyžádal povolení u krále Přemysla Otakara II.
Z toho důvodu se také o městu Třebíč hovoříme v této době jakožto o poddanském městu náležící církevní vrchnosti.
Opat dále musel vést vyšší soudy, či rozhodoval o manství\footnote{nápravníci, lidé disponující svobodným lénem}.
Opět se jednalo o strategický krok, města totiž disponovala různými privilegii, jako právo trhu či právo hradební, právem vářečným již disponovala Staržečska\footnote{Jiné označení Starého města, dnes čtvrť v Třebíči}.
Prví písemná zmínka o městu Třebíč pochází z roku 1277, v níž opat Martin oceňuje lokátora Heřmana\footnote{c. d. \bibentry{Uhlir1978}: 27}, který město Třebíč založil.

\subsubsection{Přestavba kláštera}

Během 13. století dochází k rozsáhle přestavbě kláštera.
Byly vztyčeny definitivní klášterní budovy a roku 1260 byla dokončena stavba klášterního chrámu.
Společně s přestavbou budov bylo přestavěn a vylepšen i hradební systém okolo kláštera.
Bohužel během Bitvy o Třebíč\footnote{12. květen — 9. červen 1468, součást česko-uherských válek (Druhá válka husitská)} byla většina staveb zničena či poškozena.
Postupem času vznikaly i další budovy patřící klášteru rozeseté po celém klášterem spravovaném území.

\chapter{Pozdější vývoj města Třebíče}

Město Třebíč se témeř okamžitě dostalo do problémů, díky bitvě na Moravském poli (26.srpen 1278) zavládla na Moravě nestabilní vláda a rostla moc menších knížat.
Tito knížata poté pořádala loupeživé výpravy se snahou zmocnit se královského majetku, s tím že ohroženo byla i samotný Třebíč. % 38
Situaci zachránil až nový král Jan Lucemburský, který sám Třebíč dvakrát navštívil.
Město poté dále prosperovalo a jeho význam pomalu přerostl i samotný klášter.
Výrazně pozdějšímu vývoji pomohlo privilegium markraběte Karla (Karel IV), díky kterému město získalo pevný právní řád.

Do dalších problémů se dostává město po smrti Jana Jindřicha (\CrossOpenShadow 1378) díky konfliktům Joště a Prokopa, kteří usilovali o kontrolu nad Moravou.
Tyto konflikty vyústily až v jakousi domácí válku, což vyvrcholilo koncem 14. století v těžkou hospodářskou krizi.
Situaci uklidnil až král Václav IV. roku 1413, doba klidu však neměla dlouhého trvání, protože již roku 1415 se začal stupňovat boj proti církvi, díky kázání mistra Jana Husa. % 43

\subsubsection{Třebíč a hustiská okupace}

O samotném vývoji husitství v Třebíči nemáme žádné prameny, je však známo, že Třebíč udržovala vztahy se moravskými šlechtici, jež byli myšlence nakloněni.
Během Žižkova tažení na Moravu se však třebíčský klášter dostal pod kontrolu husitů, z kláštera poté podnikaly výboje dál do okolí.
Roku 1430, však Prokop Veliký město dobývá zpět, konečně se situace uklidnila až v roce 1435.
Během hustiských válek došlo k výraznému hospodářskému úpadku, ze kterého se klášter dostával celá desetiletí. % 49-54

\subsubsection{Třebíč během Česko-uherských válek}

Třebíč měl po husitských válkách zájem o mír a byl tedy věrný králi Jiřímu z Poděbrad.
Již 12. května 1468 se Třebíč dostala do obležení, a o dva dny později byla Matyášovými vojsky dobyta.
Další boje se zaměřily na samotný klášter, ten obývalo asi čtyři tisíce lidí.
6. června se česká vojska pokusila klášter zachránit, ale bez úspěchu, 15. června se posádka v klášteře definitivně vzdala.
Po celých sedm let se bylo město opuštěné. % 54 - 56

\subsubsection{Vláda Perštejnů}

Na obnovách se začalo pracovat již roku 1474, s tím že nové město začalo vzrůstat na troskách starého.
Celková restaurace včetně klášterního velkostatku však nebyla možná díky obrovským dluhům, jenž klášter čelil.
Nejvíce pomohl Vilém z Perštejna\footnote{bohatý rod, ve službách Jagellovců, třebíčský statek je pod jeho kontrolou}, ten zavedl rybnikářství a započal opravy samotného kláštera.
Roku 1507 Vilém dělí své území mezi syny, ačkoliv apeluje ve své závěti na nutnost oprav různých klášterů v okoli o tom třebíčském nepadla ani zmíňka, existence benediktínů mu bránila v převedení třebíče do dědičného vlastnictví.
Déle se Třebíče ujímá Jan z Perštejna, který pokračuje v otcově politice, a na krátkou dobu město přesouvá pod nadvládu Jana Jetřicha.
Samotný Jan z Perštena přiděluje městu další práva, z nichž je nejdůležitější právo mílové, které značně zlepší ekonomické postavení města.
Obnoven byl na žádost městké rady i vypovídací dekret Jana Jetřicha Černohorského, s cílem omezit konkurence schopnost židovské populace ve městě.
Po Janu(\CrossOpenShadow 1548) se vlády ujímá syn Vratislav, ten si již Třebíč převádí do dědičného vlastnictví. % 61 - 68/71
% režijní feudální velkostatek

\subsubsection{Vláda Osovců}
Mezi polovinou 16. století a rokem 1613 je Třebíč državou Osovců, kdy Burian Osovský odkupuje Třebíčské panství.
Vzhledem k ceně třebíčského panství se však Burian zadlužil, dluh se snažil vykompenzovat produkcí piva a souběžně omezoval práva města -- právo na pronájem, ale i právo mílové.
Město proto podalo stížnost na zemský soud, na neuspěla, a tak se Třebíč odvolala k císaři.
Ferdinant I. poté 7. března 1560 obnovuje platnost všech privilégií, zároveň napomíná samotného Buriana, ten již dále nepodnikal tak razantní kroky proti městu.

Roku 1568 se vlády nad městem ujímá Smil Osovský z Doubravice\footnote{Jeden z důležitých pramenů doby, své paměti si zapisoval do Codexu Dubraviciana}

\chapter*{Závěr}
\addcontentsline{toc}{chapter}{Závěr}

Cíl práce se i přes problémy podařilo splnit.
Boužel se mi díky epidemii CoVid--19 nepodařilo město navštívit a získat tak mé osobní fotografie.
Ty však tvoří jen malou část práce a během nočního přejezdu jsem ve městě zastavil a alespoň prošel.

Jediný problém ke zpracování představovalo období dějin města od revolučního roku 1848 po současnost.
Bohužel v tomto časovém rozsahu nejsou ani publikace ohledně samotného Třebíče.
Byla sice plánována kniha Dějiny města III., k jejímu vytvoření snad nikdy nedošlo, či vzniklo jen pár výtisků.
Obecně nedostatek středoškolskému studentu přístupných pramenů komplikuje popsat dějiny židovské komunity v Třebíči podrobně.

I tak je výsledkem této práce obstojný souhrnný pohled na dějiny židovské obce ve Třebíči.
Sám jsem díky této práci získal hluboký náhled do historie nejen samotného Židovského města, ale i celého Třebíče a Moravy.
Pevně doufám, že se tohoto tématu ujme schopný historik a celé dějiny zmapuje podrobně od začátku do konce.


\include{literatura}

\listoffigures
\openright
\end{document}
